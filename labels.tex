\documentclass[8pt]{extarticle}

\usepackage[utf8]{inputenc}
\usepackage[T2A]{fontenc}
\usepackage[a4paper,landscape,margin=5mm]{geometry}
\usepackage{xcolor}
\usepackage{mathtools}
\usepackage{datatool}
\usepackage[many]{tcolorbox}

\pagestyle{empty}
\DTLloaddb{data}{data.csv}

\newcount\cardnrow
\newcount\cardncol

\cardncol=5
\cardnrow=6

\newlength{\cardwidth}
\newlength{\cardheight}
\newlength{\cardmargin}
\newlength{\marklength}
\newlength{\markspacing}
\newlength{\markwidth}

\setlength{\marklength}{4mm}
\setlength{\markspacing}{1mm}
\setlength{\cardmargin}{1mm}
\setlength{\markwidth}{.5pt}

\cardwidth=\dimexpr\textwidth-2.0\markspacing-2.0\marklength\relax
\divide \cardwidth by \cardncol
\advance \cardwidth by \dimexpr-2.0\cardmargin\relax

\cardheight=\dimexpr\textheight-2.0\markspacing-2.0\marklength\relax
\divide \cardheight by \cardnrow
\advance \cardheight by \dimexpr-2.0\cardmargin\relax

%\cardncol=\numexpr(\textwidth-2*(\markspacing+\markwidth))/(\cardwidth+2*\cardmargin)\relax
%\cardnrow=\numexpr(\textheight-2*(\markspacing+\markwidth))/(\cardheight+2*\cardmargin)\relax

\newcount\ter
\newcount\iter
\newcount\jter
\newcount\temp

\def\zzz{red!5!white}

\begin{document}
\ter=0\relax
\loop
{%
\noindent
\smash{\rule[-.5\markwidth]{\marklength}{\markwidth}}%
\hspace{\markspacing}%
\clap{\rule[\markspacing]{\markwidth}{\marklength}}%
%
\iter=0\relax
\loop
\advance \iter by 1\relax
	\hspace{\dimexpr\cardwidth+2\cardmargin\relax}%
	\clap{\rule[\markspacing]{\markwidth}{\marklength}}%
\ifnum \iter<\cardncol%
\repeat%
\hspace{\markspacing}%
\smash{\rule[-.5\markwidth]{\marklength}{\markwidth}}\\[\dimexpr\cardmargin-1pt\relax]
%
\iter=0\relax%
\loop 
\advance \iter by 1\relax
{%
	\raisebox{-\cardmargin}[0pt][0pt]{\rule[-0.5\markwidth]{\marklength}{\markwidth}}%
	\hspace{\markspacing}%
	\jter=0\relax%
	\loop 
	\advance \jter by 1\relax
	{%
		\hspace{\cardmargin}%
		\temp=\numexpr\jter+\cardncol*(\iter-1)+\ter\relax
		\ifnum \numexpr\temp-1\relax<\DTLrowcount{data}
			\DTLgetvalue{\EPrefix}{data}{\temp}{\dtlcolumnindex{data}{Prefix}}%
			\DTLgetvalue{\ETitle}{data}{\temp}{\dtlcolumnindex{data}{Title}}%
			\DTLgetvalue{\ESuffix}{data}{\temp}{\dtlcolumnindex{data}{Suffix}}%
			\DTLgetvalue{\EDescription}{data}{\temp}{\dtlcolumnindex{data}{Description}}%
			\DTLgetvalue{\EPackage}{data}{\temp}{\dtlcolumnindex{data}{Package}}%
			\DTLgetvalue{\ECount}{data}{\temp}{\dtlcolumnindex{data}{Count}}%
			\DTLgetvalue{\EFrame}{data}{\temp}{\dtlcolumnindex{data}{Frame}}%
			\DTLgetvalue{\EBack}{data}{\temp}{\dtlcolumnindex{data}{Back}}%
			\begin{tcolorbox}[enhanced,text fill,left=1mm,right=1mm,top=1mm,bottom=1mm,colback=\EBack,colframe=\EFrame,width=\cardwidth,height=\cardheight,before=\noindent,after=\noindent,sharp corners=all,space to upper,middle=1mm,bottomtitle=1mm,toptitle=1mm,boxsep=0mm,fonttitle=\bfseries\sffamily\huge,halign title=left, halign=left,valign=top,valign lower=center,boxrule=1pt,segmentation style={solid,line width=1pt},title={\LARGE\EPrefix}\ETitle{\LARGE\ESuffix}]%
			{\LARGE\sffamily\EDescription\par}
			\vfill
			\tcbline
			{\footnotesize\sffamily\EPackage\hfill\textbf{\ECount}}
			\end{tcolorbox}%			
	    \else
	    	\textcolor{black}{\rule{\cardwidth}{\cardheight}}%
	    	%\tcboxfit[width=\cardwidth,height=\cardheight,before=\noindent,after=\noindent,sharp corners=all]{%
			%\number\numexpr\jter+\cardncol*(\iter-1)+\ter\relax%
			%}%
	    \fi			
		\hspace{\cardmargin}%		
	}%	
	\ifnum \jter < \cardncol
	\repeat
	\hspace{\markspacing}%
	\ifnum \iter<\numexpr\cardnrow\relax
		\raisebox{-\cardmargin}[0pt][0pt]{\rule[-0.5\markwidth]{\marklength}{\markwidth}}\\[\dimexpr2.0\cardmargin-1pt\relax]
	\else
		\raisebox{-\cardmargin}[0pt][0pt]{\rule[-0.5\markwidth]{\marklength}{\markwidth}}\\[\dimexpr\cardmargin-1pt\relax]
	\fi
}%
\ifnum \iter<\cardnrow
\repeat
%
\iter=0
\hspace*{\dimexpr\marklength+\markspacing\relax}%
\rule{0pt}{\dimexpr\marklength+\markspacing\relax}%
\clap{\rule{\markwidth}{\marklength}}%
\loop
\advance \iter 1
	\hspace{\dimexpr\cardwidth+2\cardmargin\relax}%
	\clap{\rule{\markwidth}{\marklength}}%
\ifnum \iter<\cardncol%
\repeat%
}%
\advance \ter by \numexpr\cardnrow*\cardncol\relax
\ifnum \ter < \DTLrowcount{data}
\\%
\repeat%
\end{document}